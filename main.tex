\documentclass{amsart}
\usepackage{egen, egenengelsk}
\title{MAT3250 -- Discrete Mathematics}
\begin{document}
	\maketitle
	\section{Introduction}
	Discrete mathematics is the study of finite sets and structures on them.
	In this course, there are three main parts.
	In the first part, we develop techniques for counting finite sets (e.g.~``In how many ways can we partition a set of $n$ elements into $k$ non-empty subsets''?).
	In the second, we study \textbf{graphs}, which are finite sets of points and edges between them.
	In the third, we consider various algebraic structures one can define on certain finite sets (e.g.~addition and multiplication), and applications of these to coding theory and cryptography.
	
	\section{Counting}
	We are interested in counting problems, a prototypical example of which is ``Given a set $S$ of $n$ elements, how many $k$-element subsets does $S$ have?''.
	As we know, this question is answered by the binomial coefficient
	\[
	\binom{n}{k} = \frac{n \cdot (n-1) \dots (n-k+1)}{k \cdot (k-1) \cdots 2 \cdot 1} = \frac{n!}{k!(n-k)!}.
	\]
	So e.g.~, from a squad of 25 footballers, there is exactly 
	\[
	\binom{25}{11}
	\]
	ways to field a team of 11 players.
	
	We will repeat the proof of this later, after we have seen the four main tools we have for solving counting problems.
	
	\subsection{Notation}
	For any finite set $A$, we write $|A|$ for the cardinality (i.e.~the number of elements) of $A$.
	
	\subsection{Summation}
	Recall that two sets $A$ and $B$ are called \textbf{disjoint} if they have no elements in common, that is if $A \cap B = \varnothing$.
	The first ``rule'' says that if $A$ and $B$ are disjoint, then $|A \cup B| = |A| + |B|$.
	This feels pretty obvious, and it is probably more or less how you first leanred what the symbol $+$ even means. 
	We will almost always use this without mentioning it, but let's consider two examples.
	\begin{nex}
		Let $A = \{1, 3, 5\}, B = \{2, 10\}$, then $A \cap B = \varnothing$ and $A \cup B = \{1, 2, 3, 5, 10\}$.
		We have $5 = |A \cup B| = |A| + |B| = 3 + 2$.
	\end{nex}
	
	More generally, given sets $A_1,\dots, A_n$ which are \textbf{pairwise disjoint}, meaning $A_i \cap A_j = \varnothing$ for all $i \not= j$, then
	\[
	|A_1 \cup \dots \cup A_n| = \sum_{i=1}^n |A_i|.
	\]

	\begin{nex}
		Consider a group $S$ of 367 people.
		Then at least two of these will have the same birthday.
		
		\textit{Proof:} There are (including 29 Feb) 366 possible birthdays.
		For $d = 1, \dots, 366$, let
		\[
		S_d = \{x \in S \mid x\text{ has birthday on the }d\text{th day of the year}\}.
		\]
		Then the $S_d$ are all pairwise disjoint, and $S = \bigcup_{d=1}^{366} S_d$, so
		\[
		|S| = \sum_{d=1}^{366} |S_d|.
		\]
		If no two people have the same birthday, then $|S_d| \le 1$ for all $d$, which means
		\[
		367 = |S| = \sum_{d=1}^{366} |S_d| \le \sum_{d=1}^{366} 1 = 366,
		\]
		which is absurd.
		Hence there must be two people with the same birthday.
	\end{nex}
	
	\subsection{Products}
	If $A, B$ are sets, then the \textbf{product set} $A \times B$ is the set consisting of all pairs $(a,b)$ with $a \in A, b \in B$.
	The product rule for counting says that if $A$ and $B$ are finite sets, then $|A \times B| = |A| \cdot |B|$.
	Again, this is almost the definition of the multiplication operation:
	\begin{nex}
		Let $A = \{1, 2, 3\}$ and $B = \{1, 2\}$, then $A \times B = \{(1,1),(1,2),(2,1),(2,2),(3,1),(3,2)\}$, and we have
		\[
		|A \times B| = 6 = 3 \cdot 2 = |A| \cdot |B|.
		\]
	\end{nex}
If $A_1,\dots, A_n$ are sets, then $A_1 \times \dots A_n$ is the set consisting of all elements
\[
(a_1,\dots, a_n)\ \ \ \ \ \ a_i \in A_i.
\]
The product rule extends to this case: If $A_1,\dots, A_n$ are finite sets, then
\[
|A_1 \times \dots \times A_n| = |A_1|\cdots |A_n| = \prod_{i=1}^n |A_i|.
\]
\begin{nex}
	\label{ex:restaurant}
Suppose a restaurant has a menu with a set of 3 starters, 4 main courses, and 3 desserts.
What is the number of possible three course meals?
Writing $S, M$ and $D$ for the sets of starters, mains and desserts, respectively, a three course meal can be thought of as an element of $S \times M \times D$, i.e. a triple $(s,m,d)$ with $s$ a starter, $m$ a main and $d$ a dessert.
Thus the number of meals possible is 
\[
|S \times M \times D| = |S||M||D| = 3 \cdot 4 \cdot 3 = 36.
\]
\end{nex}
	
\subsection{Bijections}
If $A$ and $B$ are sets, then a \textbf{bijection} between $A$ and $B$ is, informally, a way to match up elements of $A$ and $B$ in such a way that every element of $A$ corresponds to a unique element of $B$ and vice versa.
More formally, a bijection is a function $\phi \colon A \to B$ such that for every $b \in B$ there is a \textit{unique} $a \in A$ such that $\phi(a) = b$.

The first way to show that $\phi$ is a bijection is to show that 
\begin{nlemma}
	Let $A$ and $B$ be sets, and let $\phi \colon A \to B$ and $\psi \colon B \to A$ be functions such that for all $a \in A$ we have $\psi(\phi(a)) = a$, and for all $b \in B$ we have $\phi(\psi(b)) = b$.
	Then $\phi$ and $\psi$ are bijections.
\end{nlemma}
\begin{proof}
	We prove that $\phi$ is a bijection, the case of $\psi$ is similar.
	Let $b \in B$.
	We must show that there is a unique $a \in A$ such that $\phi(a) = b$.
	If we take $a = \psi(b)$, then $\phi(a) = \phi(\psi(b)) = b$, by assumption.
	It remains to show that $a$ is unique, i.e.~that if $a'\in A$ is such that $\phi(a') = b$, then in fact $a' = a$.
	Suppose $\phi(a') = b$.
	Then
	\[
	a = \psi(b) = \psi(\phi(a')) = a',
	\]
	so $a = a'$, which is what we wanted.
\end{proof}

\begin{nex}
	Let $\ZZ$ be the set of integers and $2\ZZ$ the set of even integers.
	Then $\phi \colon \ZZ \to 2\ZZ$ given by $\phi(m) = 2m$, and $\psi \colon 2\ZZ \to \ZZ$ given by $\psi(n) = n/2$ have the property that $\phi(\psi(n)) = n$ and $\psi(\phi(m)) = m$ for all $m,n$, so define bijections between $\ZZ$ and $2\ZZ$.
\end{nex}

\begin{nlemma}
	A function $\phi \colon A \to B$ is a bijection if and only if
	\begin{itemize}
		\item $\phi$ is injective: If $a_1 \not= a_2 \in A$, then $\phi(a_1) \not= \phi(a_2)$
		\item $\phi$ is surjective: If $b \in B$, there is some $a \in A$ such that $\phi(a) = b$.
	\end{itemize}
\end{nlemma}
\begin{proof}
	Exercise.
\end{proof}
Thus the problem of showing that a given map is a bijection can often be divided into two parts, showing injectivity and surjectivity.
\todo{Change to ``rule of equality''}
The third counting principle says the following: If $A$ and $B$ are finite sets and there is some bijection between $A$ and $B$, then $|A| = |B|$.

\begin{nex}
	Consider a regular polygon in the plane (a triangle, square, etc.), with a set of corners $C$ and a set of edges $E$.
	There is a map $\phi \colon C \to E$ defined by, for every corner $c$, letting $\phi(c)$ be the edge directly adjacent to $c$ in a clockwise direction.
	This is a bijection (how would you describe the inverse map $\psi \colon E \to C$?), and so $|C| = |E|$, the number of corners equals the number of edges.
\end{nex}
	
\begin{nremark}
	The principle of bijection already snuck in in other places, e.g.~in Ex.~\ref{ex:restaurant} the set of three course meals is not exactly $S \times M \times D$, but there is an obvious bijection between the set of three course meals and $S \times M \times D$, which is good enough for counting.
\end{nremark}

\subsection{Correspondences}

\section{Fundamental counting coefficients \& permutations}

\subsection{Permutations}
\begin{ndefn}
	A \textbf{permutation} of a finite set is
\end{ndefn}

Number of permutations

Different notations for permutations

Cycles etc.

Stirling numbers of the first kind

\section{Recurrence relations}
\subsection{Various formulas and techniques for binomial coefficients}

\subsection{Formulas for the Stirling numbers of the first and second kinds}

\section{Generating functions I}

\subsection{Generating functions in general}

\subsection{Fibonacci-like sequences}
\end{document}